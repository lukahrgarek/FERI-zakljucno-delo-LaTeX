\chapter{Glavni del}
Glavni del je sestavljen iz več poglavij s smiselno izbranimi naslovi. V njem storimo naslednje:
\begin{itemize}
    \item podrobneje opišemo obravnavani problem ter predstavimo, kako se bomo lotili dela (način rešitve problema) in
    \item podrobno opišemo reševanje problema in rezultate (s pomočjo modelov, formul, algoritmov, programov, fizične izvedbe …).
\end{itemize}    
Reševanje problema pogostokrat vsebuje eksperimentalni del. V zaključnem delu podamo podrobnosti o uporabljenih sredstvih in metodah, tako da je eksperimente mogoče ponoviti in dobiti podobne rezultate. Standardnih metod (npr. statističnih) ne opisujemo podrobno. Če so metode opisane v lahko dostopni literaturi, jih citiramo in opišemo samo njihovo načelo. Druge metode in spremembe metod opišemo.
 
Opis rezultatov je bistveni del zaključnega dela. Prikažemo samo glavne, neizpodbitne rezultate, brez ponavljanja. Pišemo v preteklem času, jasno in natančno, po logičnem zaporedju (ne po zaporedju resničnega dela). Opis rezultatov naj spremlja ali pa mu sledi diskusija. V njej opišemo pomen posameznih rezultatov. Tu ni dobro ponavljati rezultatov drugih avtorjev, kakor tudi ne lastnih vmesnih rezultatov. Pomembno je opozoriti na tiste ugotovitve, ki odpirajo nova, še neraziskana področja.